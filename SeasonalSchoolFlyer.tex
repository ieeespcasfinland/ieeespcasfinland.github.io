\documentclass[12pt,epsf]{article}
\usepackage{fancybox}
\usepackage[usenames,dvipsnames,svgnames,table]{xcolor}


\topmargin0cm
\headheight0cm
\headsep0cm
\textheight26.5cm
%\textwidth18cm
\textwidth17.9cm
%\hoffset-2.5cm
%\hoffset-2.1cm
%\hoffset-2.4cm
%\hoffset-2.2cm
\hoffset-2.2cm
%\voffset-3cm
\voffset-2.7cm
\parindent.2cm
%\parskip1mm
\pagestyle{empty}
\def\defequal{\stackrel{\triangle}{=}}
\def\ban{\begin{eqnarray*}}
\def\ean{\end{eqnarray*}}
\usepackage{fancybox,framed}
\usepackage{tikz}
\usepackage{hyperref}







\begin{document}
	\vspace*{10mm}

%\thisfancypage{%
%		\setlength{\fboxsep}{5pt}\doublebox
%	}{}
\begin{framed}
\begin{center}
\Large{\bf IEEE SPS Seasonal School}\\[.3cm]
\fboxrule0.3mm
		\fbox{\parbox{13cm}{\vspace*{.0cm}
		\begin{center}
\LARGE {\bf  \textcolor{blue}{Networked Federated Learning:}}\\[.3cm]
\large{\bf Theory, Methods and Applications}
%{\sc Communications Engineers}
\vspace*{0cm} 
\end{center}
}}\\[.4cm]
\large{\bf 28 March 2022 - 01 April 2022}\\ 
%% [-.03cm]
%% {\bf (phone:\ +43 1 58801 38963; e-mail:\ franz.hlawatsch@tuwien.ac.at)}\\ 
[.2cm]
{\bf virtual in zoom. free registration (\href{https://events.vtools.ieee.org/event/register/297208}{click here}); school site: (\href{https://ieeespcasfinland.github.io/}{click here})} \\[3mm]
{\bf Organizers: IEEE Finland Chapter SP/CAS, IEEE Finland Chapter CSS/RAS/SMCS, IEEE Vizag Bay Chapter COMSOC/SPS  }
%% Seminar Room 
%SEM 389 (formerly 118)}\\[.1cm]
%%{\bf First class: March 1998, date and place to be announced}\\[.1cm]
\rule{10cm}{.3mm} \vspace*{-0cm}
\end{center}

%% \renewcommand{\baselinestretch}{1.06}\small\normalsize
\renewcommand{\baselinestretch}{.95}\small\normalsize
\vspace*{-4mm}
\large

{\bf Abstract.}
This school teaches basic theory and practical algorithms for networked federated learning (FL) from networked data. Networked data arises in several important application domains such as pandemics or the industrial internet of things. The school consists of the following modules: {\bf Machine Learning (ML)}; {\bf Networks}; {\bf Basic FL}; {\bf Clustered FL}; {\bf Trustworthy FL}. Each module consists of lectures and coding assignments in {\bf Python notebooks}.  

\vspace*{2mm}
{\bf Prerequisites.} We expect participants to have basic skills in Python programming (basic flow control, functions, numpy-arrays) and linear algebra (concept of vectors, matrices and their eigenvalues). 

\vspace*{2mm}
{\bf Learning Outcomes.} After completing the school, participants
\begin{itemize} 
\item are familiar with regularized empirical risk minimization (RERM) \\[-9mm]
\item understand the principle of gradient descent for solving RERM \\[-9mm]
\item can use graphs to represent networked data and ML models \\[-9mm]
\item can apply and critically evaluate FL methods
\end{itemize}



%{\bf References.} 
%\begin{itemize} 
%\item  Cui, S., Hero, III, A., Luo, Z., & Moura, J. (Eds.). (2016). Big Data over Networks. Cambridge: Cambridge University Press. 
%\item S. Boyd, N. Parikh, E. Chu, B. Peleato, and J. Eckstein, "Distributed Optimization and Statistical Learning via the Alternating Direction Method of Multipliers," Foundations and Trends in Machine Learning, 3(1):1–122, 2011.
%\item D. A. Spielman, "Spectral and Algebraic Graph Theory," Incomplete Draft, Yale University, 2019 
%\item A. Jung, "Federated Learning Over Networks for Pandemics," LiveProject, Manning Publishers, 2021 
%\item A. Jung, "Machine Learning: The Basics," Springer, Singapore, 2022. 
%\end{itemize} 

{\bf Acknowledgment.} This seasonal school is supported by the IEEE Signal Processing Society, Aalto University, the TalTech Industrial project (European Union’s Horizon 2020 research and innovation programme under grant agreement No 952410), and the Academy of Finland (project ‘‘Intelligent Techniques in Condition Monitoring of Electromechanical Energy Conversion Systems,’’ decision number 331197). 

\vspace*{10mm}
\includegraphics[height=0.07\paperheight]{SPS_Logo_Color_RGB}
\hspace*{10mm}
\includegraphics[height=0.07\paperheight]{horizon2020}
\hspace*{10mm}
\includegraphics[height=0.07\paperheight]{aaltologo}
\hspace*{10mm}
\includegraphics[height=0.07\paperheight]{AoFLogo}
\vspace*{2mm}




\end{framed}

\end{document} %%%%%%%%%%%%%%%%%%%%%






  



 


